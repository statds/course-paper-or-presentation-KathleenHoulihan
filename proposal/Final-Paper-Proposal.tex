\documentclass[12pt]{article}
\usepackage{amsmath}
\usepackage[margin = 1in]{geometry}
\usepackage{graphicx}
\usepackage{booktabs}
\usepackage{natbib}
\usepackage{float}




\usepackage{lipsum}

\usepackage[colorlinks=true, citecolor=blue]{hyperref}



\title{UCSAS 2024 USOPC Data Challenge}
\author{Kathleen Houlihan\\
  Department of Statistics\\
  University of Connecticut
}

\begin{document}
\maketitle

\begin{abstract}

  At the Summer Olympics in Paris in 2024, ninety-six men and ninety-six women 
  from around the world will compete on various apparatus. Twelve teams of five athletes will be 
  featured in the team events for both men's and women's events. In terms of events, women
  will compete on four apparatus while men will compete on six apparatus. This paper along 
  with my senior honors thesis will focus on the completion of the UCSAS 2024 USOPC Data Challenge. 
  The UCSAS 2024 USOPC Data Challenge tasks individuals with developing an analytics model that
  can be used to identify and compare the expected medal count for the United States male and 
  female artistic gymnasts at the 2024 Summer Olympics in Paris. At the Olympics there are 8 
  medal events for men consisting of team all-around, individual all-around, floor exercise, 
  pommel horse, still rings, vault, parallel bars, and high bar and 6 medal events for the women 
  consisting of team all-around, individual all-around, vault, uneven bars, balance beam, and 
  floor exercise. 

\end{abstract}

\section{Introduction}
\label{sec:intro}

The Summer and Winter Olympics are held every four years, traditionally in a unique country and city.
 The importance of the Olympics cannot be understated as the games are a symbol of peaceful global 
 interaction and give people hope that a better world is possible. For decades, gymnastics has been 
 the most watched sport in the Summer Olympics and the United States has been known for bringing gold 
 medal-winning gymnasts to compete. In the 2020 Summer Olympics games in Tokyo, the United States 
 female artistic gymnastics team took home the gold medal in the Women's All Around, a bronze medal 
 in the Women's Balance Beam, a gold medal in the Women's Floor Exercise, a silver medal for the Women's
 Team, a bronze medal in the Women's Uneven Bars, and a silver medal in the Women's Vault. Noting that 
 the United States was the only country to have a team medal in all six women's artistic gymnastics 
  categories, the likely hood of the United States bringing female athletes that will medal in the 2024 
  Paris Olympics is very probable. However, in the 2020 Summer Olympics games in Tokyo, the United States 
  male artistic gymnastics team did not medal at all. This data challenge not only tasks individuals 
  with predicting the expected medal count, but also predicting which athletes the United States will 
  and should select to bring to the Paris 2024 Summer Olympics in order to maximize the number of medals 
  the Unites States teams win. 

\\

The rest of this proposal is organized as follows. First, I will identify the data that is available
 in order to complete this project in Section~\ref{sec:data}. Next, I will briefly discuss my intended 
 analytical methods in Section~\ref{sec:meth}. Lastly, I will analyze the literature that is already 
 available on this topic in Section~\ref{sec:lit}.

\section{Data}
\label{sec:data}

This year, the University of Connecticut Sports Analytic Symposium (UCSAS) and the United States 
Olympic and Paralympic Committee has released a data challenge focused on identifying a group of 
five athletes who will enable the United States Men's and Women's Artistic Gymnastics teams to 
maximize earned medals at the Paris 2024 Summer Olympics. This paper will serve as a demonstration
of my preliminary work towards completing this data challenge. To predict which of the United States 
athletes are most likely to medal on the various apparatuses, I will be using the cleaned data that 
is provided in the UCSAS data challenge that includes data from major domestic and international 
gymnastic competitions from 2022 and 2023. The cleaned data provides the last name, first name, gender, 
country, date, competition, round, location, apparatus, rank, difficulty score, execution score, penalty, 
and overall score of various potential Olympic athletes at various competitions. This data is available, 
public, and already cleaned to be easily loaded into R studio where the majority of the computations 
required of this project will be completed.

\section{Methods}
\label{sec:meth}

The proposed method of this paper will involve calculating and examining the mean and standard 
deviations of each American male and female athlete's score on each apparatus and then using the 
mean score as the main predictor and the standard deviation to determine if an athlete with a lower 
mean is more consistent in cases where mean scores are similar. Then, we will use the mean score and 
standard deviations to determine which athletes are ranked in the top ten on each apparatus. Using 
these top 10 athletes on each apparatus we will produce combinations of five athletes and calculate 
each combination's expected additive score. Using this information we will be able to select which 
five athletes for each gender the United States will be most likely to select to bring to the Olympics 
based on both mean and standard deviation. Furthermore, once the predicted United States teams have 
been selected we will be able to compare the mean scores and standard deviation of each United States 
athlete on each apparatus to the mean scores and standard deviation of each athlete from other countries 
in order to predict the leader board for each apparatus. This information can then be used to predict 
team and individual overall by examining the average of each top athlete's scores on each apparatus. 
From these computations we will be able to predict the leader board and medal distribution at the Paris 
2024 Summer Olympics.

\section{Literature Review}
\label{sec:lit}

Currently there is no literature available relevant to predicting gymnastics outcomes at the 2024 
Paris Summer Olympics. However, there is literature that relates to various aspects of the data that 
will be used to form our predictions. For example, the article \textit{The Prediction of All-Around 
Event Final Score Based On D and E Score Factors In Women's Artistic Gymnastics} aims to determine the 
impact of individual apparatus D and E score in artistic gymnastics in relation to the final result of 
all-around event which could be useful in allowing us to understand how D and E scores, which are 
available in the data, play an effort in predicting the overall score. Additionally, the paper 
\textit{Performance rating in men’s world elite artistic gymnastics: a status-quo study on scoring 
tendencies at Olympic Games following rule changes} also can play a similar role in predicting overall 
score. This paper's objective is to analyze the effect of the 2006 rule change which aimed to reduce the 
dominance of difficulty while strengthening performance differentiation  in elite artistic gymnastics. 
Notably, the paper found that the rule change resulted in significantly enhanced performance 
differentiation with the execution or E score being the main predictor of the final score as opposed to 
the difficulty or D score. Furthermore, in the article \textit{How Apparatus Difficulty Scores Affect All 
Around Results in Men's Artistic Gymnastics} the authors use regression, cluster, and ANOVA analysis to 
analyze how difficulty scores predict overall score differently across different apparatuses. Ultimately, 
the available literature on predicting artistic gymnastic success provides a lot of information on how we 
can use the E and D scores provided in the data sets to predict overall score in addition to just looking 
at mean score and standard deviation.

\begin{thebibliography}{999}
  \bibitem{D and E score Predictions}
  Almir Atiković, Edina Kamenjašević \emph{Science of Gymnastics Journal}
 The Prediction of All-Around Event Final Score Based On D and E Score Factors In Women's Artistic 
 Gymnastics
\bibitem{Olympic Scoring Tendencies}
  Jonas Rohleder, Alexandra Pizzera, Jonas Breuer, Tobias Vogt \emph{Taylor and Francis Online}
  Performance rating in men’s world elite artistic gymnastics: a status-quo study on scoring tendencies 
  at Olympic Games following rule changes.
\bibitem{Difficulty Scoring}
  Ivan Čuk, Warwick Forbes \emph{Science of Gymnastics Journal}
  How Apparatus Difficulty Scores Affect All Around Results in Men's Artistic Gymnastics.
\bibitem{Data Challenge} 
  UCSAS 2024 USOPC Data Challenge,
 UCSAS Data Challenge \emph{Data Challenge}.
   The goal of the data challenge is to identify the group of 5 athletes who will enable the Team USA 
   Olympic Men’s and Women’s Artistic Gymnastics teams to optimize success in Paris 2024.
\end{thebibliography}
\bibliography{refs}
\bibliographystyle{mcap}

\end{document}
