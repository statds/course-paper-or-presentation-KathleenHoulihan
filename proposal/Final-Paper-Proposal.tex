\documentclass[12pt]{article}
\usepackage{amsmath}
\usepackage[margin = 1in]{geometry}
\usepackage{graphicx}
\usepackage{booktabs}
\usepackage{natbib}
\usepackage{float}




\usepackage{lipsum}

\usepackage[colorlinks=true, citecolor=blue]{hyperref}



\title{UCSAS 2024 USOPC Data Challenge}
\author{Kathleen Houlihan\\
  Department of Statistics\\
  University of Connecticut
}

\begin{document}
\maketitle

\begin{abstract}

  At the Summer Olympics in Paris in 2024, ninety-six men and ninety-six women 
  from around the world will compete on various apparatus. Twelve teams of five athletes will be 
  featured in the team events for both men's and women's events. In terms of events, women
  will compete on four apparatus while men will compete on six apparatus. This paper along 
  with my senior honors thesis will focus on the completion of the UCSAS 2024 USOPC Data Challenge. 
  The UCSAS 2024 USOPC Data Challenge tasks individuals with developing an analytics model that
  can be used to identify and compare the expected medal count for the United States male and 
  female artistic gymnasts at the 2024 Summer Olympics in Paris. At the Olympics there are 8 
  medal events for men consisting of team all-around, individual all-around, floor exercise, 
  pommel horse, still rings, vault, parallel bars, and high bar and 6 medal events for the women 
  consisting of team all-around, individual all-around, vault, uneven bars, balance beam, and 
  floor exercise. 

\end{abstract}

\section{Introduction}
\label{sec:intro}

The Summer and Winter Olympics are held every four years, traditionally in a unique country and city.
 The importance of the Olympics cannot be understated as the games are a symbol of peaceful global 
 interaction and give people hope that a better world is possible. For decades, gymnastics has been 
 the most watched sport in the Summer Olympics and the United States has been known for bringing gold 
 medal-winning gymnasts to compete. In the 2020 Summer Olympics games in Tokyo, the United States 
 female artistic gymnastics team took home the gold medal in the Women's All Around, a bronze medal 
 in the Women's Balance Beam, a gold medal in the Women's Floor Exercise, a silver medal for the Women's
 Team, a bronze medal in the Women's Uneven Bars, and a silver medal in the Women's Vault. Noting that 
  the United States was the only country to have a team medal in all six women's artistic gymnastics 
  categories, the likely hood of the United States bringing female athletes that will medal in the 2024 
  Paris Olympics is very probable. However, in the 2020 Summer Olympics games in Tokyo, the United States 
  male artistic gymnastics team did not medal at all. This data challenge not only tasks individuals 
  with predicting the expected medal count, but also predicting which athletes the United States will 
  and should select to bring to the Paris 2024 Summer Olympics in order to maximize the number of medals 
  the Unites States teams win. 

\\

The rest of this proposal is organized as follows. First, I will identify the data that is available
 in order to complete this project in Section~\ref{sec:data}. Next, I will briefly discuss my intended 
 analytical methods in Section~\ref{sec:meth}. Lastly, I will analyze the literature that is already 
 available on this topic in Section~\ref{sec:lit}.

\section{Data}
\label{sec:data}



\section{Methods}
\label{sec:meth}



\section{Literature Review}
\label{sec:lit}



\begin{thebibliography}{999}

\end{thebibliography}
\bibliography{refs}
\bibliographystyle{mcap}

\end{document}
