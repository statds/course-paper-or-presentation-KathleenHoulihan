\documentclass[12pt]{article}
\usepackage{amsmath}
\usepackage[margin = 1in]{geometry}
\usepackage{graphicx}
\usepackage{booktabs}
\usepackage{natbib}
\usepackage{float}




\usepackage{lipsum}

\usepackage[colorlinks=true, citecolor=blue]{hyperref}



\title{UCSAS 2024 USOPC Data Challenge}
\author{Kathleen Houlihan\\
  Department of Statistics\\
  University of Connecticut
}

\begin{document}
\maketitle

\begin{abstract}

  At the Summer Olympics in Paris in 2024, ninety-six men and ninety-six women 
  from around the world will compete on various apparatus. Twelve teams of five athletes will be 
  featured in the team events for both men's and women's events. In terms of events, women
  will compete on four apparatus while men will compete on six apparatus. This paper along 
  with my senior honors thesis will focus on the completion of the UCSAS 2024 USOPC Data Challenge. 
  The UCSAS 2024 USOPC Data Challenge tasks individuals with developing an analytics model that
  can be used to identify and compare the expected medal count for the United States male and 
  female artistic gymnasts at the 2024 Summer Olympics in Paris. At the Olympics there are 8 
  medal events for men consisting of team all-around, individual all-around, floor exercise, 
  pommel horse, still rings, vault, parallel bars, and high bar and 6 medal events for the women 
  consisting of team all-around, individual all-around, vault, uneven bars, balance beam, and 
  floor exercise. 

\end{abstract}

\section{Introduction}
\label{sec:intro}



\section{Data}
\label{sec:data}



\section{Methods}
\label{sec:meth}



\section{Literature Review}
\label{sec:lit}



\begin{thebibliography}{999}

\end{thebibliography}
\bibliography{refs}
\bibliographystyle{mcap}

\end{document}
