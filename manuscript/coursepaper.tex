\documentclass[12pt]{article}
\usepackage{amsmath}
\usepackage[margin = 1in]{geometry}
\usepackage{graphicx}
\usepackage{booktabs}
\usepackage{natbib}
\usepackage{float}




\usepackage{lipsum}

\usepackage[colorlinks=true, citecolor=blue]{hyperref}



\title{UCSAS 2024 USOPC Data Challenge}
\author{Kathleen Houlihan\\
  Department of Statistics\\
  University of Connecticut
}

\begin{document}
\maketitle

\begin{abstract}

  At the Summer Olympics in Paris in 2024, ninety-six men and ninety-six women 
  from around the world will compete on various apparatus. Twelve teams of five athletes will be 
  featured in the team events for both men's and women's events. In terms of events, women
  will compete on four apparatus while men will compete on six apparatus. This paper along 
  with my senior honors thesis will focus on the completion of the UCSAS 2024 USOPC Data Challenge. 
  The UCSAS 2024 USOPC Data Challenge tasks individuals with developing an analytics model that
  can be used to identify and compare the expected medal count for the United States male and 
  female artistic gymnasts at the 2024 Summer Olympics in Paris. At the Olympics there are 8 
  medal events for men consisting of team all-around, individual all-around, floor exercise, 
  pommel horse, still rings, vault, parallel bars, and high bar and 6 medal events for the women 
  consisting of team all-around, individual all-around, vault, uneven bars, balance beam, and 
  floor exercise. In this paper, I will identify which five male and female artistic gymnasts 
  are best suited to be sent to the Paris 2024 Olympics on behalf of the United States in order 
  to maximize the expected medal count.

\end{abstract}

\section{Introduction}
\label{sec:intro}

The Summer and Winter Olympics are held every four years, traditionally in a unique country and city.
 The importance of the Olympics cannot be understated as the games are a symbol of peaceful global 
 interaction and give people hope that a better world is possible \citep{United Nations}. For decades, gymnastics has been 
 the most watched sport in the Summer Olympics and the United States has been known for bringing gold 
 medal-winning gymnasts to compete. In the 2020 Summer Olympics games in Tokyo, the United States 
 female artistic gymnastics team took home the gold medal in the Women's All Around, a bronze medal 
 in the Women's Balance Beam, a gold medal in the Women's Floor Exercise, a silver medal for the Women's
 Team, a bronze medal in the Women's Uneven Bars, and a silver medal in the Women's Vault \citep{Olympics}. 
 Noting that the United States was the only country to have a team medal in all six women's artistic gymnastics 
  categories, the likely hood of the United States bringing female athletes that will medal in the 2024 
  Paris Olympics is very probable. However, in the 2020 Summer Olympics games in Tokyo, the United States 
  male artistic gymnastics team did not medal at all. This data challenge not only tasks individuals 
  with predicting the expected medal count, but also predicting which athletes the United States will 
  and should select to bring to the Paris 2024 Summer Olympics in order to maximize the number of medals 
  the Unites States teams win. 

\\

The rest of this paper is organized as follows. First, I will identify the data that is available
 in order to complete this project in Section~\ref{sec:data}. Next, I will discuss my 
 analytical methods to determine the atheltes that should be sent to the Olympics from the United States and how 
 I am calculating the anticipated medal count in Section~\ref{sec:meth}. Then, I will present the results of my 
 methods in Section~\ref{sec:res}. Lastly, I will discuss the meaning of my results, some challenges, 
 and some limitations of this project in Section~\ref{sec:dis}.

\section{Data}
\label{sec:data}

This year, the University of Connecticut Sports Analytic Symposium (UCSAS) and the United States 
Olympic and Paralympic Committee has released a data challenge focused on identifying a group of 
five athletes who will enable the United States Men's and Women's Artistic Gymnastics teams to 
maximize earned medals at the Paris 2024 Summer Olympics. To predict which of the United States 
athletes are most likely to medal on the various apparatuses, I will be using the cleaned data that 
is provided in the UCSAS data challenge that includes data from major domestic and international 
gymnastic competitions from 2022 and 2023. The cleaned data provides the last name, first name, gender, 
country, date, competition, round, location, apparatus, rank, difficulty score, execution score, penalty, 
and overall score of various potential Olympic athletes at various competitions. This data is available, 
public, and already cleaned to be easily loaded into R studio where the majority of the computations 
required of this project will be completed. 
\\
The UCSAS data challenge also provides data from major domestic
and international gymnastic competions from 2017 to 2021 which must be kept separate from the more recent 2022 
and 2023 data due to the fact that the Code of Points scoring system is changed each Olympic cycle. Regardless 
of this fact, the primary data source for this paper will be the data from 2022 and 2023, as time, injuries, 
and other factors can have a large impact on the success of a gymnast, so the greatest predictor
of the Olympic outcomes in 2024 will be the most recent data.

\section{Methods}
\label{sec:meth}

The first portion of this project involves determining which 5 female and male United States athletes
should be brought to the 2024 Olympics in order to maximize success. For women, this means determining 
which five athletes stand the greatest chance of winning the gold medal for the women's individual all-around,
team all-around, Balance Beam, Floor Exercise, Uneven Bars, and Vault. For men, this means determining
which five athletes stand the greatest chance of winning the gold medal for men's individual all-around, team 
all-around, Floor exercise, Horizontal Bar, Parallel Bars, Pommel Horse, Rings, and Vault. 

In order to select the appropriate athletes, I first calculated and examined the mean and standard 
deviations of each American male and female athlete's score on each apparatus. Then, I plotted the mean 
versus the standard deviation of the ten athletes with the highest mean scores on each apparatus. 
Using this visual, I created a parameter for each apparatus that uses both mean score and standard 
deviation to identify which athletes are best suited to be considered to compete on behalf of the United 
States on each apparatus. Using this parameter, I identified the best five athletes that have competed on
each apparatus. 

Next, using only the athletes I have selected using the parameters, 
I calculated the sum of each athletes mean score for each apparatus and the sum of each athlete's standard 
deviation for each apparatus. Similarly to before, I plotted the mean versus the standard deviation for the 
ten athletes with the highest summed mean score from these selected athletes. This plot is intended to represent
the United States' best candidates for earning a gold medal for the individual all-around component of the Olympics.

Then, using the sum of each athletes mean score for each apparatus and the sum of each athlete's standard 
deviation for each apparatus which was calculated above, I calculated the expected score and standard deviation sum 
of the total number of atheltes on our limited list chose three, which represents the possible total scores
for different all-around team combinations.

Finally, in order to build a final team of five, we will include the best three candidates for the team 
all-around, the best candidate for the individual all-around, and manually examine if there is a candidate that 
is superior on a paricular apparatus that should be included.


\section{Results}
\label{sec:res}



\section{Discussion}
\label{sec:dis}

One of the most challenging parts of this research is that there are no prior studies that have similar
aims to predict olympic success or candidates to model the design and methods off of. Additionally, 
the data challenge source provides so much information that determining which data is most valuable 
can be challenging. Currently, our model is limited as it does not take into account the 
Execution score, Difficulty score, penalty, or importance of specific competitions or rounds over others. 
Furthermore, another limitation of our model is that we generally tried to avoid selecting athletes that
had very few attempts on a particular apparatus even if their few or singular scores were exceptional.

\begin{thebibliography}{999}
  \bibitem{D and E score Predictions}
  Almir Atiković, Edina Kamenjašević \emph{Science of Gymnastics Journal}
 The Prediction of All-Around Event Final Score Based On D and E Score Factors In Women's Artistic 
 Gymnastics
\bibitem{Olympic Scoring Tendencies}
  Jonas Rohleder, Alexandra Pizzera, Jonas Breuer, Tobias Vogt \emph{Taylor and Francis Online}
  Performance rating in men’s world elite artistic gymnastics: a status-quo study on scoring tendencies 
  at Olympic Games following rule changes.
\bibitem{Difficulty Scoring}
  Ivan Čuk, Warwick Forbes \emph{Science of Gymnastics Journal}
  How Apparatus Difficulty Scores Affect All Around Results in Men's Artistic Gymnastics.
\bibitem{Data Challenge} 
  UCSAS 2024 USOPC Data Challenge,
 UCSAS Data Challenge \emph{Data Challenge}.
   The goal of the data challenge is to identify the group of 5 athletes who will enable the Team USA 
   Olympic Men’s and Women’s Artistic Gymnastics teams to optimize success in Paris 2024.
\bibitem{United Nations}
   The Olympic Movement, the United Nations and the Pursuit of Common Ideals
   \emph{UN Chronicle}
   The global importance of the Summer and Winter Olympics.
\bibitem{Olympics}
   Tokyo 2020 Artistic Gymnastics Results
   \emph{Olympic Games Tokyo 2020}
   The results of the Tokyo 2020 Summer Olympic artistic gymnastics.
\end{thebibliography}
\bibliography{refs}
\bibliographystyle{mcap}

\end{document}